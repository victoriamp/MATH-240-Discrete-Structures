\documentclass{article}
\usepackage[utf8]{inputenc}
\usepackage{geometry}
\geometry{margin = .25in}
\usepackage{amsmath, amsthm, amssymb, enumitem}
\usepackage{etoolbox}
\newtheorem{problem}{Problem}
\theoremstyle{definition}
\newtheorem*{solution}{Solution}

\begin{document}

\title{Math 240 (Discrete Structures 1) Assignment 3}
\author{Victoria Pittard, ID 260762268}
\date{\today}

\maketitle

%%Problem 1
\begin{problem}
\textbf{Proofs with sets:} Let A, B, C be arbitrary sets. For each of the following statements, either prove it is true (without a Venn diagram) or give a counterexample to show that it is false.
\begin{enumerate}[label = \alph*)]
    \item $(A \setminus B)\setminus C = A\setminus (B \cup C)$
    
    \item $(A\oplus B = A \oplus C$ and $A \cap B = A \cap C) \implies B \implies C$
    
    \item $(A \cup B) \times (C \cup D) = (A \times C) \cup (B \times D)$
\end{enumerate}
\end{problem}

\begin{solution}
\end{solution}
\begin{enumerate}[label=\alph*)]
    \item
    $(A\setminus B) \setminus C$ includes the elements that are in set A, but the elements in B are removed, then those in C are removed. Thus, it contains the elements in A, but neither B nor C.\\
    $A\setminus (B\cup C)$\\
    $(B\cup C)$ is the union of sets B and C, including the elements that are in B, C, or their intersection.\\
    Thus,  $A\setminus (B\cup C)$ includes the elements of A, but the elements in B and/or C are removed. It therefore contains elements in A, but neither B nor C.\\
    The first specification contains all elements of the second, and the second specification contains all elements of the first, making them equivalent.
    
    \item 
    $A\oplus B$ is equivalent to the union of sets A and B, minus their intersection. Likewise, $A\oplus C$ is equivalent to the union of sets A and C, minus their intersection. \\
    $A\cap B$ is the intersection of A and B. Likewise, $A\cap C$ is the intersection of A and C. \\
    If both the B's union with A (without the intersection) and C's union with A (without the intersection) are equal, and B's intersection with A and C's intersection with C are also equal, they share the same elements with A,  and have the same elements that A does not, thus implying that B=C.
    
    \item
       $(A\cup B) \times (C \cup D) \equiv (A \times C) \cup (A \times D) \cup (B \times C) \cup (B \times D)$ (Distributive)\\
       $(A \times C) \cup (A \times D) \cup (B \times C) \cup (B \times D) \neq (A\times C) \cup (B\times D)$
       A counterexample to this statement is where A contains 1, 2, 3; B contains 2, 3, 4; C contains 3, 4, 5; D contains 4, 5, 6\\
       In this case, the left side of the equation works out to be the set of coordinate pairs (x, y) formed when x = {2, 3}, y = {4, 5}, and the right side of the equation works out to be the set of pairs (x1, y1) or (x2, y2) formed when x1 = {1, 2, 3}, y1 = {3, 4, 5}, x2 = {2, 3, 4}, y2 = {4, 5, 6}. It is evident that there is a much wider range of coordinate pairs in the statement on the right than in the statement on the left for this example, thus disproving the assumption they are equal with a counterexample.
\end{enumerate}

%Problem 2
\begin{problem}
\textbf{Relations:}
\begin{enumerate}[label = \alph*)]
    \item Determine whether or not each relation is reflexive, symmetric, antisymmetric, and/or transitive. For each property, if the relation has that property, prove it. If it doesn't have that property, give a counterexample. State if the relation is a total order, partial order but not a total order, or neither; justify your answer.
    \begin{enumerate}
        \item $\mathcal{R} = \{(X,Y) \in (\mathcal{P}(A))^2 | X \cap Y \neq \emptyset \}$ where A is some arbitrary set
        
        \item $\mathcal{R} = \{(a,b) \in \mathbb{N}^2 | a divides b\}$ (a divides b means that there is some integer k such that b = ka)
    \end{enumerate}
    
    \item For $a,b \in \mathbb{R} \setminus {0}$, define a~b iff $a\setminus b \in \mathbb{Q}$. Prove that ~ defines an equivalence relation on $\mathbb{R}\setminus{0}$. Show that $[(9-\sqrt{5})/(1-\sqrt{5}]$ = $[2/(3-6\sqrt{5}]$.
\end{enumerate}
\end{problem}

\begin{solution}
\end{solution}

\begin{enumerate}[label = \alph*)]
    \item
    \begin{enumerate}
        \item 
        $\mathcal{R} = \{(X, Y) \in \mathcal{P}(A)^2 | X \cap Y \neq \emptyset\}$\\
        \underline{Reflexive}: yes\\
            $\forall X \in \mathcal{P}(A)^2, X \cap X = X$     (Idempotent)\\
            $\therefore X \cap X \neq \emptyset$ \\
            $\therefore (X,X) \in \mathcal{R}$\\
        \underline{Symmetric}: yes\\
            $(X, Y) \in \mathcal{P}(A)^2 \implies X\cap Y \neq \emptyset$\\
            $X\cap Y = Y\cap X$ (Commutative)\\
            $\therefore Y\cap X \neq \emptyset$\\
            $\therefore Y\cap X \in \mathcal{R}$\\
        \underline{Transitive}: no\\
            $(X,Y)\in \mathcal{P}(A)^2 \implies X\cap Y \neq \emptyset$\\
            $(Y,Z)\in \mathcal{P}(A)^2 \implies Y\cap Z \neq \emptyset$\\
            \textit{Counterexample:}\\
                Let $X=\{1, 2\}, Y = \{2, 3\}, Z = \{3, 4\}$\\
                $X\cap Y = \{2\} \neq \emptyset, Y\cap Z = \{3\} \neq \emptyset$\\
                $X\cap Z = \emptyset$, as there is no overlap between {1, 2} and {3, 4}.
        \underline{Antisymmetric}: no\\
            $(X,Y)\in \mathcal{P}(A)^2 \implies X\cap Y \neq \emptyset$\\
            $(Y,X) \in \mathcal{P}(A)^2 \implies Y\cap X\neq 
            \emptyset$\\
            \textit{Counterexample:}\\
                Let $X = \{1, 2\}, Y = \{2, 3\}$\\
                $X\nsubseteq Y, Y \nsubseteq X$\\
                $\therefore X\neq Y$\\
        \underline{Conclusion:} As shown above, the relation is reflexive and symmetric. As it is neither transitive nor antisymmetric, it cannot be a partial or total order relation.
        
        \item
        $\mathcal{R} = \{(a, b) \in \mathbb{N}^2 | a divides b$ such that b = ka\\
        \underline{Reflexive}: yes\\
            $\forall x \in \mathbb{N}, x = kx$ whenever $k=1$\\
            $1\in \mathbb{Z}$\\
            $\therefore a divides a$\\
            $\therefore a \in \mathcal{R}$\\
        \underline{Symmetric}: no\\
            $(x, y) \in \mathbb{N}^2 \implies x divides y \implies y=kx$\\
            Thus, $\frac{1}{k}y = x$\\
            $\therefore,$ y divides x, making (y,x) part of the relation, if and only if $\frac{1}{k}\in \mathbb{Z}$\\
            \textit{Counterexample:}\\
            $k=2 \implies y = 2x \implies x = \frac{1}{2}y$\\
            $\frac{1}{2}\notin \mathbb{Z} \therefore (y,x) \notin \mathbb{N}^2$\\
        \underline{Transitive}: yes\\
        $(x, y) \in \mathbb{N}^2 \implies y = kx$\\
        $(y,z) \in \mathbb{N}^2 \implies z=ly$\\
        Where both k and m are integers.\\
        From this, it can be concluded that $\frac{z, l} = y$\\ $\frac{z}{l}=kx$\\
        $z=(kl)x$
        As the product of two integers (k and l) is also an integer, $z=mx$ where m is the integer product of k and l.\\
        This implies that x divides z, and $(x, z) \in \mathbb{N}^2$\\
        \underline{Antisymmetric}: yes\\
        $(x, y) \in \mathbb{N}^2 \implies y = kx$\\
        $(y,x) \in \mathbb{N}^2 \implies x=my$\\
        Where both k and m are integers.\\
        From this, it can be concluded that $\frac{x}{l}=y \implies x=(kl)x$\\
        This is only true when $kl=1, \implies x = 1x \implies x=x$\\
        This means that $x=y$ or else $k, l \neq 1$ or $k, l \notin \mathbb{Z}$\\
        \underline{Totality}: no\\
        $\forall a, b \in \mathbb{N}^2$ either (a, b) or (b,a) is true.\\
        \textit{Counterexample:}\\
        Let $a=2, b=3$\\
        $3=2k \implies k=\frac{3}{2} \notin \mathbb{Z}$\\
        $2=3k \implies k=\frac{2}{3} \notin \mathbb{Z}$\\
        \underline{Conclusion:}\\
        As shown above, the relation is reflexive, transitive, and antisymmetric, making it a partial order relation. Totality is not satisfied, and it is not a total order relation. It is not symmetric.
    \end{enumerate}
    
    \item $\mathcal{R} = \{a, b \in \mathbb{R}\setminus \emptyset | \frac{a}{b} \in \mathbb{Q} \implies a \sim b\}$\\
    \underline{Reflexive:} yes\\
    $\forall a \in \mathbb{R}\setminus \emptyset, \frac{a}{a} = 1\in \mathbb{Q}$\\
    $\therefore a, a \in \mathcal{R}$\\
    \underline{Symmetric:} yes\\
    $a, b \in \mathbb{R}\setminus \emptyset \implies \frac{a}{b} \in \mathbb{Q}$\\
    $\implies a = kb, where k\in \mathbb{Z}$ by the Euclidean Algorithm.\\
    As k is an integer, $\frac{1}{k} \in \mathbb{Q}$\\
    $\therefore \frac{1}{k} = m, ma = b \implies \frac{b}{a} \in \mathbb{Q}$\\
    $\therefore b, a \in \mathcal{R}$\\
    \underline{Transitive:} yes\\
    $a, b \in \mathbb{R} \setminus \emptyset \implies a=kb, k\in \mathbb{Q}$\\
    $b, c \in \mathbb{R} \setminus \emptyset \implies b=mc, m\in \mathbb{Q}$\\
    $a=(km)c$\\
    As m and k are both integers, their product (km) is also an integer).\\
    $\therefore b, c \in \mathcal{R}$\\
    \underline{Conclusion:} As shown above, the relation, is reflexive, symmetric, and transitive, $\sim$ represents an equivalence relation.
    \\
    \underline{Equivalence Classes}\\
    $let \frac{9 - \sqrt{5}}{1-\sqrt{5}} = a, \frac{2}{3-6\sqrt{5}} = x$\\
    (left to right)\\
    Suppose [x] = [a]\\
    Since $x\in [X]$ (Reflexive) $\implies x\in[A]$\\
    $\implies x\sim a$\\
    (right to left)\\
    Suppose $x\sim a$. We will show $[x]\subseteq [a] (a)$ and $[a]\subseteq [x] (b)$\\
    (a) Let $y\subseteq [x] \implies y\sim x$\\
    $y\sim x, x\sim a \implies y \sim a$ (transitivity)\\
    $y\sim a\implies y \in [a] \implies [x] \subseteq [a]$\\
    (b) Let $z\in [a] \implies z \sim a$\\
    (since $x\sim a$) $z\sim x \implies z \in [x]$\\
    $\\implies [a] \subseteq [x]$\\
    Together, these statements prove it from right to left, as (a) and (b) imply that [a] = [x]
\end{enumerate}

%Problem 3
\begin{problem}
\textbf{Proof techniques:} Prove the following statements using the method of your choice (direct proof, proof of the contrapositive, proof by contradiction).
\begin{enumerate}[label = \alph*)]
    \item Let $a,b\in \mathbb{R}$. If $a \in \mathbb{Q}$ and $b\notin \mathbb{Q}$, then $a +- b \notin \mathbb{Q}$
    
    \item If the average of 4 distinct integers is 10, then at least one of the integers is greater than 11.
\end{enumerate}
\end{problem}

\begin{solution}
\end{solution}

\begin{enumerate}[label = \alph*)]
    \item 
    $a, b, c, d \in \mathbb{Z}$; $a\neq b\neq c\neq d$\\
    $\frac{a+b+c+d}{4} = 10 \implies a+b+c+d = 40$\\
    To the contrary, assume that a,b,c,d are all less than or equal too 11. Thus, each can be represented as (11-xi), where i ranges from 1-4 and all values of integers xi are greater than or equal to zero.\\
    $11-x1+11-x2+11-x3+11-x4=40$\\
    $44-(x1+x2+x3+x4) = 40$\\
    For the above to be true, either at least one x is less than zero, or there are repeated values of x, thus contradicting the idea that a,b,c,d are unique integers.\\
    As further evidence, the smallest sum of unique values of the x's is $0+1+2+3 = 6$, and 6 is greater than 4\\
    Proof by contradiction.
    
    \item 
    \underline{Addition}\\
    Assume $x\notin \mathbb{Q}$ and $x + \frac{a}{b} = \frac {c}{d}$, where $a,b,c,d \in \mathbb{Z}$ (or: $\frac{a}{b}), \frac{c}{d} \in \mathbb{Q}$, and $b,d\neq 0$\\
    $x + \frac{a}{b} = \frac {c}{d}\implies x= \frac{c}{d} - \frac{a}{b}\implies x=\frac{cb-ad}{bd}$\\
    Integers are closed under multiplication and subtraction, thus x is an integer (CONTRADICTION).\\
    \underline{Subtraction}\\
    Assume $x\notin \mathbb{Q}$ and $x - \frac{a}{b} = \frac {c}{d}$, where $a,b,c,d \in \mathbb{Z}$ (or: $\frac{a}{b}), \frac{c}{d} \in \mathbb{Q}$, and $b,d\neq 0$\\
    $x-\frac{a}{b} = \frac{c}{d} \implies x= \frac{c}{d} + \frac{a}{b} \implies x=\frac{bc+ad}{bd}$\\
    Integers are closed under multiplication and addition, thus x is an integer (CONTRADICTION).
\end{enumerate}

\end{document}
