\documentclass{article}
\usepackage[utf8]{inputenc}
\usepackage{geometry}
\geometry{margin = .25in}
\usepackage{amsmath, amsthm, amssymb, enumitem}
\usepackage{etoolbox}
\newtheorem{problem}{Problem}
\theoremstyle{definition}
\newtheorem*{solution}{Solution}

\begin{document}

\title{Math 240 (Discrete Structures 1) Assignment 1}
\author{Victoria Pittard, ID 260762268}
\date{}

\maketitle

%%Problem 1
\begin{problem}
Use a truth table to determine if each statement is a tautology, contradiction.
\end{problem}

\begin{enumerate} [label=\alph*)]

\item $(P \vee Q) \Rightarrow \neg P$

\item $(P \Leftrightarrow Q) \wedge (Q \Leftrightarrow R) \Rightarrow (P \Leftrightarrow R)$

\item $[(P \oplus Q) \oplus \neg Q] \Leftrightarrow P$

\end{enumerate}


\begin{solution}
\end{solution}

\begin{enumerate} [label = \alph*)]

\item  
\begin{tabular}{|c|c|c|c|c|}
    \hline
    P & Q & $P \vee Q$ & $\neg P$ & $(P \vee Q) \Rightarrow \neg P$ \\ \hline
    T & T & T & F & F \\ \hline
    T & F & T & F & F \\ \hline
    F & T & T & T & T \\ \hline
    F & F & F & T & T \\ \hline
\end{tabular}

Contingency
\item
\begin{tabular}{|c|c|c|c|c|c|c|c|}
    \hline
     P & Q & R & $P \Leftrightarrow Q$ & $Q \Leftrightarrow R$ & $P \Leftrightarrow R$ & $(P \Leftrightarrow Q) \wedge (Q \Leftrightarrow R)$ & $(P \Leftrightarrow Q) \wedge (Q \Leftrightarrow R) \Rightarrow (P \Leftrightarrow R)$\\ \hline
     T & T & T & T & T & T & T & T \\ \hline
     T & T & F & T & F & F & F & T \\ \hline
     T & F & T & F & F & T & F & T \\ \hline
     T & F & F & F & T & F & F & T \\ \hline
     F & T & T & F & T & F & F & T \\ \hline
     F & T & F & F & F & T & F & T \\ \hline
     F & F & T & T & F & F & F & T \\ \hline
     F & F & F & T & T & T & T & T \\ \hline
\end{tabular}

Tautology 
\item
\begin{tabular}{|c|c|c|c|c|c|}
    \hline
    P & Q & $P \oplus Q$ & $\neg Q$ & $(P \oplus Q) \oplus \neg Q$ & $[(P \oplus Q) \oplus \neg Q] \Leftrightarrow P$ \\ \hline
    T & T & F & F & F & F \\ \hline
    T & F & T & T & F & F \\ \hline
    F & T & T & F & T & F \\ \hline
    F & F & F & T & T & F \\ \hline
\end{tabular}

Contradiction
\end{enumerate}

%%Problem 2
\begin{problem}
Verify the following statements using only identities (see the list posted on MyCourses). Show all of your work and name the identity or identities used in each step.
\end{problem}

\begin{enumerate}[label = \alph*)]
\item $[(P \Rightarrow Q) \wedge P] \Rightarrow Q$ is a tautology

\item $\neg (P \wedge Q) \wedge (Q \Rightarrow P) \equiv \neg Q$

\item $\neg [(P \vee Q) \vee [(Q \vee \neg R) \wedge (P \vee R)]] \equiv \neg P \wedge \neg Q$
\end{enumerate}

\begin{solution}
\end{solution}

\begin{enumerate}[label = \alph*)]
    \item 
    \begin{align*}
        [(P \Rightarrow Q) \wedge P] \Rightarrow Q &\equiv [(\neg P \vee Q) \wedge P] \Rightarrow Q   &\text{(Conditional)} \\
         &\equiv [P \wedge (\neg P \vee Q)] \Rightarrow Q  &\text{(Commutative)}    \\
         &\equiv [(P \wedge \neg P) \vee (P \wedge Q)] \Rightarrow Q &\text{(Distributive)} \\
         &\equiv [FF \vee (P \wedge Q)] \Rightarrow Q &\text{(Complement)} \\
         &\equiv (P \wedge Q) \Rightarrow Q &\text{(Identity)} \\
         &\equiv \neg (P \wedge Q) \vee Q &\text{(Conditional)} \\
         &\equiv (\neg P \vee \neg Q) \vee Q &\text{(DeMorgan's)} \\
         &\equiv \neg P \vee (\neg Q \vee Q) &\text{(Associative)} \\
         &\equiv \neg P \vee T &\text{(Complement)} \\
         &\equiv T &\text{(Domination)}
    \end{align*}
    Therefore, $[(P \Rightarrow Q) \wedge P] \Rightarrow Q$ is a tautology (always true).
   
    \item
    \begin{align*}
        \neg (P \wedge Q) \wedge (Q \Rightarrow P) &\equiv \neg Q \\
        &\equiv (\neg P \vee \neg Q) \wedge (Q \Rightarrow P) &\text{(DeMorgan's)} \\
        &\equiv (\neg P \vee \neg Q) \wedge (\neg Q \vee P) &\text{(Conditional)} \\
        &\equiv (\neg Q \vee \neg P) \wedge (\neg Q \vee P) &\text{(Commutative)} \\
        &\equiv \neg Q \vee (\neg P \wedge P) &\text{(Distributive)} \\
        &\equiv \neg Q \vee F &\text{(Complement)} \\
        &\equiv \neg Q &\text{(Identity)} \\
        \neg Q &\equiv \neg Q
    \end{align*}
    
    \item
    \begin{align*}
        \neg [(P \vee Q) \vee [(Q \vee \neg R) \wedge (P \vee R)]] &\equiv \neg P \wedge \neg Q \\
        &\equiv \neg[(P\vee Q) \vee [(Q \wedge P) \vee (Q \wedge R) \vee (\neg R \wedge P) \vee (\neg R \wedge R)] &\text{(Distributive)}\\
        &\equiv \neg[(P \vee Q) \vee [(P \wedge Q) \vee (Q \wedge R) \vee (\neg R \wedge P) \vee F]] &\text{(Commutative, Complement)}\\
        &\equiv \neg [(P \vee Q) \vee (P \wedge Q) \vee (Q \wedge R) \vee (\neg R \wedge P)] &\text{(Associative, Identity)}\\
        &\equiv \neg[(P\vee Q)\vee (Q \wedge R) \vee (\neg R \wedge P)] &\text{(Absorption)}\\
        &\equiv \neg(P\vee Q)\wedge \neg(Q\wedge R)\wedge \neg(\neg R \wedge P) &\text{(DeMorgan's)}\\
        &\equiv (\neg P \wedge \neg Q) \wedge (\neg Q \vee \neg R) \wedge (R \vee \neg P) &\text{(DeMorgan's)}\\
        &\equiv (\neg P \wedge \neg Q) \wedge (R \vee \neg P) &\text{(Idempotent)}\\
        &\equiv (\neg P \wedge \neg Q) &\text{(Idempotent)}\\
        \neg P \wedge \neg Q &\equiv \neg P \wedge \neg Q &\text{(Associative)}
    \end{align*}
\end{enumerate}

%%Problem 3
\begin{problem}
Of the following conditional and biconditional statements, which are true and which are false? Briefly justify your answers.
\begin{enumerate}[label = \alph*)]
    \item $\pi$ is an integer if and only if $\sqrt{e+3}$ is a vowel.
    
    \item $0>1$ whenever $2+2=4$
    
    \item If (a) implies (b), then pigs cannot fly.
\end{enumerate}
\end{problem}

\begin{solution}
\end{solution}

\begin{enumerate}[label = \alph*)]
    \item 
    P: $\pi$ is an integer; Q: $\sqrt{e+3}$ is a vowel\\
    ``if and only if" indicates a biconditional statement: $P \Leftrightarrow Q$ \\
    $pi$ is an irrational number, and therefore not an integer (P is \textbf{false}). $\sqrt{e+3}$ is also an irrational number, and therefore not a vowel (Q is \textbf{false}). \\
    In order for a biconditional statement to be true, the truth values of P and Q must be the same. As both atoms in this sentence are false (same truth value), the statement is \textbf{true}.
    
    \item 
    Q: 0>1; P: 2+2=4\\
    Whenever indicates a conditional statement: $P \Rightarrow Q$ \\
    0 is not greater than one, therefore Q is \textbf{false}. 2+2=4 is a true statement, therefore P is \textbf{true}. $P\Rightarrow Q$ is true in the event that either P is false (not satisfied) or Q is true (not satisfied). Neither condition is satisfied, so the statement is \textbf{false}.
    
    \item
    A: A, B: B, P: Pigs can fly \\
    ``If" and ``implies" both indicate conditional statements: $(A\Rightarrow B) \Rightarrow \neg P$ \\
    Pigs are unable to fly, therefore P is \textbf{false} all of the time ($\neg P$, or the right half of the conditional is \textbf{true}. However, $(A\Rightarrow B)$ cannot be proven true (as it is a conditional, not biconditional statement) or false (would be proven only if the right half of the conditional $(\neg P)$ was false, and it is true). Therefore, the truth of the sentence \textbf{cannot be determined}.
    
\end{enumerate}

%%Problem 4
\begin{problem}
Symbolize the following English sentences in logic, using the abbreviation scheme provided.
\begin{enumerate}[label = \alph*)]
    \item ``Thunder only happens when it's raining." \\
    T: thunder happens; R: it's raining
    
    \item ``For every positive integer $n$ there is a prime number that is bigger than $n$ but at most $2n$ \\
    I(x): x is a positive integer; P(x): x is a prime number; B(x,y): x is bigger than y
    
    \item ``Golbach's Conjecture is true if every even integer greater than 2 can be written as the sum of two primes." \\
    G: Goldbach's Conjecture is true; E(x): x is an even integer; T(x): x is greater than 2; P(x): x is the sum of two primes.
\end{enumerate}
\end{problem}

\begin{solution}
\end{solution}

\begin{enumerate}[label = \alph*)]
    \item 
    $R \Rightarrow T$ \\
    ``T only when R" can be rewritten as ``R implies T", indicating a conditional statement.
    
    \item
    $\forall x (I(x) \Rightarrow \exists y ( P(y) \wedge B(y, x) \wedge B(2x+1, y)))$\\
    $\forall x$ : ``for all values of x..."\\
    $(I(x)\rightarrow )$ : ``if x is an even integer, then..."\\
    $\exists y$ : ``there exists an integer y such that..."\\
    $P(y) \wedge$ : ``y is prime and..."\\
    $B(y,x) \wedge$ : ``y is greater than x and..."\\
    $B(2x+1, y)))$ : ``y is at most 2x (less than 2x+1)'' \\
    Rewritten: ``For all values of x, if x is an even integer, then there exists an integer y such that y is prime, greater than x, and less than 2x+1"
    
    \item
    $\forall x ((E(x) \wedge T(x)) \Rightarrow P(x)) \Rightarrow G$\\
    $\forall x$ : ``For all values of x..."\\
    $((E(x) \wedge T(x))$ : ``If x is an even integer greater than two..."\\
    $\Rightarrow P(x))$ : ``Implies that x can be written as the sum of two prime numbers..." \\
    $\Rightarrow G$ : ``Then Goldbach's Conjecture is true."\\
    Rewritten: ``For all values of x, if x is an even integer greater than two implies that x can be written as the sum of two numbers, then Goldbach's Conjecture is true."
\end{enumerate}

\end{document}
